\documentclass[a4paper]{article}
\usepackage{geometry}
\geometry{verbose,a4paper,tmargin=2cm,bmargin=1.5cm,lmargin=2cm,rmargin=3cm}
\usepackage{color}
\usepackage{framed}
\definecolor{shadecolor}{rgb}{0.9,0.9,0.9}
\setlength{\parskip}{\medskipamount}
\setlength{\parindent}{0pt}
\usepackage{setspace}
\usepackage{amsmath}
\onehalfspacing

\usepackage{Sweave}
\begin{document}

\title{Exploratory data analysis with FLR}
\author{Ernesto Jardim <ernesto@ipimar.pt>\\
Manuela Azevedo <mazevedo@ipimar.pt\\
IPIMAR, Av.Brasilia, 1449-006 Lisboa}
\date{}
\maketitle
Exploratory data analysis in FLR is done using the package FLEDA, mainly focus on data available for stock assessment. FLEDA was developed under the project IPIMAR/NeoMAv. It includes a combination of simple calculations and graphical representations aiming at data screening (checking for missing data, unusual values, patterns, etc), inspection of data consistency (within and between data series) and extracting signals from the basic data. Diagnostics include those recommended during the 2004 Methods Working Group meeting (ICES, 2004).

This paper uses the example data set included in FLR (North Sea plaice stock, ple4) and is structured by (i) Catch and Effort, (ii) Abundance indices, (iii) Biomass and (iv) Total mortality.

First one needs to load the required packages and data.

\begin{center}
\begin{minipage}[H]{0.95\textwidth}%
\begin{shaded}%
\begin{Schunk}
\begin{Sinput}
> require(FLEDA)
\end{Sinput}
\begin{Soutput}
FLEDA 2.0 "The Swordfish hobnobber"
------------------------------------
\end{Soutput}
\begin{Sinput}
> data(ple4)
> data(ple4sex)
> data(ple4.index)
> data(ple4.indices)
\end{Sinput}
\end{Schunk}
\end{shaded}%
\end{minipage}
\end{center}

Testing formula $x=\sum{y_i}$ \[ y=\sum{x_i} \] \begin{equation}
	z=\sum{y_i}
\end{equation}.

kkkkkkkkk

\end{document}
