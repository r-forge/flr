% kobe.Rnw --
%
% Author: laurence kell <lauriekell@gmail.com>

%\VignetteIndexEntry{kobe}
% Meta information - fill between {} and do not remove %
% \VignetteIndexEntry{An R Package for ...}
% \VignetteDepends{foo, bar, ...}
% \VignetteKeywords{kwd1, kwd2}
% \VignettePackage{...}

\documentclass[a4paper, 11pt, oldtoc]{artikel1}
\usepackage[onehalfspacing]{setspace}
\usepackage{natbib} \bibliographystyle{plain}
\usepackage{graphicx, psfrag, Sweave}
\usepackage{enumerate}


\begin{document}


%------ Frontmatter ------
\title{Kobe Plotting}
\author{Laurie Kell\footnote{ICCAT}}
\date{June 2012}
\maketitle
\tableofcontents

\section{Introduction}

Scientific Advice within the tuna Regional Fisheries Management Organisations (tRFMOs) is based on a common framework 
(ref). The advice requires estimation of the probabilities of $F \textless F_{MSY}$ and
$B$ \textgreater $B_{MSY}$ for a range of management options, generally a range of total allowable catchs (TAC). Advice is presented as 

\begin{itemize}
 \item a phase plot showing fishng mortality and stock status relevant to management reference points
 \item a pie chart summarising current stock status; and 
 \item a Kobe strategy matrix which summarises the probability of achieving management targets for different management options over time.
\end{itemize}

This requires, a stock assessment, estimates of reference points and projections
In this document we show how R can be used to summarise stock assessment results in the Kobe II format using the R package \texttt{kobe}. 

The package can be loaded with the command:


\section{Input Data}

The package requires a data.frame with columns for stock and fishing mortality relative and columns describing
any scenarios. Data can be read in from different stock assessment outputs using methods 
in other FLR packages; see the section on Stock Assessment Packages for examples.

The basic data needed are time series of $F/F_{MSY}$ and $SSB/B_{MSY}$ (or some other benchmark such as $F_{0.1}$)
for historic and projections. These need to include estimates of uncertainty, e.g. produced by bootstrapping an
assessment or running Monte Carlo Markov Chain simulations. An example data set is


\section{Stock Trends}

\section{Phase plot}

\section{Probabilities}

\section{K2SM}

\section{Assessment Packages}

\begin{Schunk}
\begin{Sinput}
> library(kobe)
> library(ggplot2)
> library(plyr)
> library(reshape)
> ## example data sets
> data(sims)
\end{Sinput}
\end{Schunk}

\begin{Schunk}
\begin{Sinput}
> ggplot(assmt)   + 
+   geom_hline(aes(yintercept=1),col="red",size=2)  + 
+   geom_line( aes(year,stock,group=iter,col=iter))
\end{Sinput}
\end{Schunk}


\begin{figure}
\begin{center}
